\chapter{Other Notions of Symmetry}
\label{chap:other_notions}

\section{Other types of round symmetry}

As we said in the beginning, we are interested in other types of symmetry too.

A stronger variation of the symmetry we described is to demand that all round scrambles be done in the same manner (i.e. permute identically). We call this \emph{uniform round symmetry}.

Conversely, a weaker notion of symmetry than the one described is what we call \emph{Parikh round symmetry}: keeping in mind that the letters in process transducers are subsets of the signal set $I$, a scramble in rounds can not only move letters but also signals, as long as the Parikh image of the round w.r.t.\! $I$ is unchanged (i.e. every $i\in I$ appears the same number of times in the round as originally).

The original notion is here called \emph{symbol-wise round symmetry}.

For all examples, denote $I=O=[m]=\{0,\cdots, m-1\}$.

\begin{example}[Parikh does not imply symbol-wise]
Let $m\geq 3$ and $\pi=(0\ 1)$. For every $k$, we construct a transducer that is length-$k$ Parikh round-equivalent, but not length-$k'$ symbol-wise round-equivalent for any $k'$.

Construct the deterministic transducer $\cT$ as follows: $\cT=\tup{I,O,S,s_0,\delta,\lab}$ where
\begin{gather*}
    S = \{ s_0, \mathrm{sink}, 
        s_1, \dots, s_{k-1}, 
        t_1, \dots, t_{k-1} \} \\
    \lab(s_1)=\{0\}, \lab(t_1)=\{1\},
        \forall s\notin\{s_1, t_1\}:\  \lab(s)=\emptyset \\
    \delta(s_0, \{0\}) = s_1,\ 
        \delta(s_0, \{1,2\}) = t_1, \\
    \delta(s_{k-1}, \{1,2\}) = s_0,\ 
        \delta(t_{k-1}, \{0\}) = s_0, \\
    \forall 1\leq i \leq k-2:\ 
        \delta(s_i, \bullet)= s_{i+1},\ 
        \delta(t_i, \bullet)= t_{i+1}, \\
    \text{All other transitions lead to $\mathrm{sink}$}
\end{gather*}
\antodo[inline]{Draw or make this more intuitive.}

It is Parikh round-symmetric since upon permuting a round of the form $\mathbf{r}=\{0\}\sigma_2\cdots\sigma_{k-1}\{1,2\}$, one can obtain $\mathbf{r}'=\{1,2\}\sigma_2\cdots\sigma_{k-1}\{0\}$ that satisfies $T(\mathbf{r}')=\pi(T(\mathbf{r}))$, such that $\mathbf{r}'\equivP \mathbf{r}$ but not necessarily $\mathbf{r}'\equivS \mathbf{r}$. \antodo{Formalize.} This also holds conversely, for rounds of the form of $\mathbf{r}'$. For all other rounds, the output is always $\emptyset^k$, so for them $T$ is symmetric.
\end{example}

\begin{example}[Round Robin]
\antodo[inline]{Show that it trivially satisfies symmetry, and call its strong symmetry \emph{uniform round-symmetry}.}
\end{example}

See `my-notes-p57-61.pdf` for further details.

\subparagraph*{Extension to round simulation}
All this could also be extended to round simulation.

\section{Symmetry in other settings}

Consider a transducer $\cT=\tup{I,O,S,s_0,\delta,\lab}$ over $I=\{i_1,\ldots,i_k\}$ and $O=\{o_1,\ldots,o_k\}$. We say that $\cT$ is \emph{Ultimately Symmetric} if for every permutation $\pi\in \cS_k$ and for every $x\in \Io$, there exists $k\ge 0$ such that $\cT(\pi(x))[k:\infty]=\pi(\cT(x))[k:\infty]$.

That is, for every word $x$, apart from some finite prefix, the output of $\cT$ on $\pi(x)$ is identical to the permuted output $\pi(\cT(x))$.

The following is a proposition we are yet to prove, but we strongly believe we are soon to succeed.

\begin{hypothesis}
Let $A,B$ be two NFAs over $\Sigma$. There exists a bound $M=M(A,B)$ that satisfies the following: let $x\in \Sigma^\star$ and $x'\equiv x$ a scramble thereof. Denote $q'\in\delta_1(q,x)$ and $s'\in\delta_2(s,x')$, then there exists such an $x$ with $|x|<M$.
\end{hypothesis}

To understand how this theorem can help our cause, recall that the algorithm for checking $k$-simulation in rounds was based on a construction of a DFA $A_1^k$ and an NFA $A_2^k$ so that, if we fix $T_1$ and $T_2$ and gradually increase $k$, we will have a sequence of automata with identical states, wherein only the transitions and the alphabet change.

If we succeed to show that for some $k_1$ and $k_2$ the two pairs of automata are identical in transitions, then by help of the theorem we can bound the difference $k_2-k_1$ and hopefully get a deterministic analysis of the values of $k$ which would work, i.e. we would solve the problem of bounding $k$.

\begin{theorem}
	The problem of deciding whether a transducer $\cT$ is ultimately symmetric can be solved in polynomial time.
\end{theorem}
\begin{proof}
	First, we observe that if $\cT$ is ultimately symmetric with respect to permutations $\pi$ and $\tau$ then it is also ultimately symmetric with respect to $\pi\circ\tau$. Thus, it is enough to check ultimate symmetry for the two generators of $\cS_k$, and we henceforth focus on deciding ultimate symmetry for a fixed permutation $\pi\in \cS_k$.
	
	We obtain from $\cT$ a deterministic co-B\"uchi automaton $\cC_{\cT,\pi}=\tup{Q,\tI,\mu,q_0,\alpha}$ as follows. Intuitively, $\cC_{\cT,\pi}$ simulates two copies of $\cT$, where the second copy is permuted by $\pi$ (i.e., when seeing the input $\vec{i}\in \tI$ it actually simulates the transition of $\cT$ with $\pi(\vec{i})$). Then, each state $(q,r)$ is marked as accepting if the permuted labelling of $q$ is the same as the labelling of $r$. 
	We then show that $\cC$ accepts a word $x\in \Io$ iff there exists $k\ge 0$ such that $\cT(\pi(x))[k:\infty]=\pi(\cT(x))[k:\infty]$, so all that remains is to decide whether $L(\cC)=\Io$, which can be done in polynomial time.
	
	\shtodo[inline]{add the details.}
	
\end{proof}

