\chapter{PSPACE Hardness}
\label{chap:apx}

\begin{lemma}
\label{lem:universalityofnfa}
Universality of \glspl{nfa} over alphabet $\Sigma=\{0,1\}$, where all states are accepting, and the degree of nondeterminism is at most $2$, is $\PSPACE$-complete.
\end{lemma}
\begin{proof}
In~\cite{kao2009nfas}, it is shown that universality of \glspl{nfa} remains $\PSPACE$-complete even for \glspl{nfa} over alphabet $\Sigma=\{0,1\}$ and all states accepting. Thus, we only need to show that this remains the case under the restriction that $|\delta(q,\sigma)|\le 2$ for every state $q$ and letter $\sigma$.

To see this, we start by observing that universality remains $\PSPACE$-complete for \glspl{nfa} over alphabet $\{0,1,\$\}$ with nondeterminism degree at most 2. Indeed, given an \gls{nfa} over $\{0,1\}$ with maximal nondeterminism degree $d>2$, we can replace each transition of the form\footnote{We can assume all transitions have degree exactly $d$ by adding redundant transitions} $\delta(q,\sigma)=\{q_1,\ldots, q_d\}$ with a binary tree of depth $\lceil \log d \rceil$, reading $\$$ on all transitions, which starts at $q$ and ends in $q_1,\ldots,q_d$. Thus, we introduce at most $d$ states for every transition. By marking these states as accepting, this reduction maintains universality, and requires a polynomial blowup.

Next, we observe that the reductions in~\cite[Lemma 2]{kao2009nfas} first transform an \gls{nfa} over alphabet size $k$ to an \gls{nfa} over alphabet size $k+1$ with all states accepting and with identical nondeterminism degree (indeed, the only added transitions are in fact deterministic), and then transforms an \gls{nfa} with all states accepting and alphabet size $4$ to an \gls{nfa} with all states accepting and alphabet size $2$, with an equal nondeterminism degree (essentially by encoding each of the 4 letters as two letters in $\{0,1\}$).

Since we start this chain of reductions with an \gls{nfa} of nondeterminism degree at most 2, we maintain this property throughout the proof.
\end{proof}

\section{Proof of Theorem~\ref{thm:equivalence_PSPACE-H}}
\label{apx:proof_equivalence_PSPACE-H}

We show a reduction from the universality problem for \glspl{nfa} over alphabet $\{0,1\}$ where all states are accepting and the degree of nondeterminism is at most 2, to round-equivalence with $k=2$ and with $\fL$ given as a \gls{dfa} of constant size. The former is shown to be $\PSPACE$-hard in~\cref{lem:universalityofnfa}.

Consider an \gls{nfa} $\cN=\tup{Q,\{0,1\},\delta,q_0,Q}$ where $|\delta(q,\sigma)|\le 2$ for every $q\in Q$ and $\sigma\in \{0,1\}$.
%Set $\Sigma=\{0,1\}$ and $\Lambda=(ab+cd)^*$, and let $A=\tup{Q, \Sigma, \delta,q_0,Q}$ be an \gls{nfa}. 
We construct two transducers $\cT_1$ and $\cT_2$ over input and output alphabets $\sI=\{a,b,c,d\}$ and $\sO=\{\top,\bot\}$ and $\fL\subseteq \sI^*$, such that $L(\cN)=\{0,1\}^*$ iff $\cT_1\equiv_{2,\fL}\cT_2$. 
%That is, iff for all $x\in \fL$ there exist $x'\req[2] x$ with $x'\in \fL$ and $x''\req[2] x$ that satisfy $\cT_1(x)\req[2] \cT_2(x')$ and $\cT_1(x'')\req[2] T_2(x)$.

Set $\fL=(ab+cd)^*$ (described as a 4-state \gls{dfa}). Intuitively, our reduction encodes $\{0,1\}$ into $\{a,b,c,d\}^2$ by setting $0$ to correspond to $ab$ and to $ba$, and $1$ to $cd$ and to $dc$. Then, $\cT_1$ keeps outputting $\top$ for all inputs in $\fL$, thus mimicking ``accepting'' every word in $\{0,1\}^*$. We then construct $\cT_2$ so that every nondeterministic transition of $\cN$ on e.g., $0$ is replaced by two deterministic branches on $ab$ and on $ba$. Hence, when we are allowed to permute $ab$ and $ba$ by round equivalence, we capture the nondeterminism of $\cN$. 

\begin{figure}[ht]
\begin{minipage}{.33\linewidth}
	\begin{tikzpicture}[shorten >=1pt,node distance=1.5cm and 1.3cm,on grid,auto]
	    \tikzstyle{state}=[circle, inner sep=0mm, outer sep=0mm, minimum size=8mm, draw=black];
 	%   \clip (-2,0.9) rectangle (2.4, -1.3);
		\node[state] [initial text={\,}] (q_0) [red, initial above] {$\top$}; 
		\node[state] (q_1) [red, right=of q_0] {$\top$}; 
		\node[state] (q_2) [red, left=of q_0] {$\top$};
		\node[state] (q_3) [red, below=of q_0] {$\bot$};
		
		\path[->] 
		(q_0)
		 edge [bend right, swap, pos=0.6] node {$a$} (q_1)
		 edge [bend left] node {$c$} (q_2)
		 edge [pos=0.7] node {$b,d$} (q_3)
		(q_1) 
		 edge [bend right, swap] node {$b$} (q_0)
		 edge [out=-90, in=0] node[pos=0.8,yshift=0.2cm] {$a,c,d$} (q_3)
		(q_2)
	     edge [bend left] node {$d$} (q_0)
		 edge [out=-90, in=180, swap] node[pos=0.8,yshift=0.2cm] {$a,b,c$} (q_3);
	\end{tikzpicture}
 	\caption{The transducer $\cT_1$ in the proof of~\cref{thm:equivalence_PSPACE-H}.}
 	\label{fig:PSPACE_reduction_T1}
 \end{minipage}%
 \hfill%
 \begin{minipage}{.63\linewidth}
	\begin{tikzpicture}[shorten >=1pt,node distance=1.5cm and 1.3cm,on grid,auto]
		\tikzstyle{state}=[circle, inner sep=0mm, outer sep=0mm, minimum size=8mm, draw=black];
		\node[state] (q_0) {$q$}; 
		\node[state] (q_1) [above right=of q_0] {$q^{0,0}$}; 
		\node[state] (q_2) [right=of q_0] {$q^{0,1}$}; 
		\node[state] (q_3) [above left=of q_0] {$q^{1,0}$}; 
		\node[state] (q_4) [left=of q_0] {$q^{1,1}$}; 
		
		\path[->] 
		(q_0)
		 edge node [pos=0.6] {$0$} (q_1)
		 edge node [pos=0.6] {$0$} (q_2)
		 edge node [pos=0.6, swap] {$1$} (q_3)
		 edge node [pos=0.6, swap] {$1$} (q_4);
	\end{tikzpicture}%
	\hspace{0.5cm}%
	\begin{tikzpicture}[shorten >=1pt,node distance=1.5cm and 1.3cm,on grid,auto]
	    \tikzstyle{state}=[circle, inner sep=0mm, outer sep=0mm, minimum size=8mm, draw=black];
		\node[state] (q_0) {$q$}; 
		\node[state, label={[font=\footnotesize,label distance=-.1cm]above:$q_a$}] (q_1) [above right=of q_0] {$\color{red} \top$}; 
		\node[state, label={[font=\footnotesize,label distance=-.1cm]above:$q_b$}] (q_2) [right=of q_0] {$\color{red} \top$}; 
		\node[state, label={[font=\footnotesize,label distance=-.1cm]above:$q_c$}] (q_3) [above left=of q_0] {$\color{red} \top$}; 
		\node[state, label={[font=\footnotesize,label distance=-.1cm]above:$q_d$}] (q_4) [left=of q_0] {$\color{red} \top$}; 
		\node[state, label={[font=\footnotesize,label distance=-.1cm]above:$q^{0,0}$}] (q_1b)[right=of q_1] {$\color{red} \top$}; 
		\node[state, label={[font=\footnotesize,label distance=-.1cm]above:$q^{0,1}$}] (q_2b)[right=of q_2] {$\color{red} \top$}; 
		\node[state, label={[font=\footnotesize,label distance=-.1cm]above:$q^{1,0}$}] (q_3b)[left=of q_3] {$\color{red} \top$}; 
		\node[state, label={[font=\footnotesize,label distance=-.1cm]above:$q^{1,1}$}] (q_4b)[left=of q_4] {$\color{red} \top$}; 
		
		\path[->] 
		(q_0)
		 edge node [pos=0.6] {$a$} (q_1)
		 edge node [pos=0.6] {$b$} (q_2)
		 edge node [pos=0.6, swap] {$c$} (q_3)
		 edge node [pos=0.6, swap] {$d$} (q_4)
		(q_1) edge node [pos=0.6] {$b$} (q_1b)
		(q_2) edge node [pos=0.6] {$a$} (q_2b)
		(q_3) edge node [pos=0.6, swap] {$d$} (q_3b)
		(q_4) edge node [pos=0.6, swap] {$c$} (q_4b);
	\end{tikzpicture}
 	\caption{Every state and its 4 transitions in $\cN$ (left) turn into 8 transitions in $\cT_2$ (right). All transitions not drawn in the right figure lead to $q_\bot$, a sink state labelled $\color{red}\bot$.}
 	\label{fig:PSPACE_reduction_T2}
 \end{minipage}
\end{figure}

We now proceed to define the reduction formally. We construct $\cT_1$ independently of $\cN$, as depicted in~\cref{fig:PSPACE_reduction_T1}, containing 4 states. For every $x\in \fL$ we have $\cT_1(x)=\top^{|x|}$, and for every other $x\notin \fL$ we have $\cT_1(x)=\top^{m}\bot^{|x|-m}$ where $m$ is the length of the maximal prefix of $x$ in $(ab+cd)^*(a+c+\epsilon)$.

We proceed to construct $\cT_2$. 
%Recall that the nondeterminism degree of $\cN$ is $2$. Thus,
We can think of the outgoing transitions from every state $q$ as $\delta(q,0)=\{q^{0,0},q^{0,1}\}$ and $\delta(q,1)=\{q^{1,0},q^{1,1}\}$ (unless $\cN$ has no outgoing transitions on one of the letters, see below). We obtain $\cT_2$ from $\cN$ by introducing 4 new states $q_a,q_b,q_c,q_d$ for every state $q\in Q$, and setting the transitions and labels as depicted in~\cref{fig:PSPACE_reduction_T2}. In case $\cN$ does not have a transition on e.g., $0$ from $q$, then instead of going to $q_a$ or $q_b$, we proceed to a new state $q_\bot$ labelled $\bot$, which is a sink state. In addition, $q_\bot$ is reached upon any transition not yet defined.
Observe that for every $x\in \fL$ we have $\cT_2(x)=\top^{m}\bot^{|x|-m}$ for some $0\le m\le |x|$ (since $q_\bot$ is a sink).
% \begin{figure}
% \begin{minipage}{.30\linewidth}%[ht]
%  	%\centering
% 	\begin{tikzpicture}[shorten >=1pt,node distance=1cm and 1.5cm,initial distance=0.2cm,on grid,auto]
% 	\tikzstyle{smallnode}=[circle, inner sep=0mm, outer sep=0mm, minimum size=5mm, draw=black];
% 	%\clip (-5,2) rectangle (5, -4);
% 	\clip (-2,0.9) rectangle (2.4, -1.3);
% 		\node[smallnode, initial text={}] (q_0) [initial above] {$\color{red} \top$}; 
% 		\node[smallnode] (q_1) [right=of q_0] {$\color{red} \top$}; 
% 		\node[smallnode] (q_2) [left=of q_0] {$\color{red} \top$};
% 		\node[smallnode] (q_3) [below=of q_0] {$\color{red} \bot$};
		
% 		\path[->] 
% 		(q_0)
% 		 edge [bend right, swap, pos=0.6] node {$a$} (q_1)
% 		 edge [bend left] node {$c$} (q_2)
% 		 edge [pos=0.7] node {$b,d$} (q_3)
% 		(q_1) 
% 		 edge [bend right, swap] node {$b$} (q_0)
% 		 edge [out=-90, in=0] node[pos=0.8,yshift=0.2cm] {$a,c,d$} (q_3)
% 		(q_2)
% 	     edge [bend left] node {$d$} (q_0)
% 		 edge [out=-90, in=180, swap] node[pos=0.8,yshift=0.2cm] {$a,b,c$} (q_3);
% 	\end{tikzpicture}
%  	\caption{The transducer $\cT_1$ in the proof of~\cref{thm:equivalence_PSPACE-H}.}
%  	\label{fig:PSPACE_reduction_T1}
%  \end{minipage}%
%  \hfill%
%  %\hspace{2.0cm}%
%  \begin{minipage}{.65\linewidth}%[ht]
% 	%\centering
% 	\begin{tikzpicture}[shorten >=1pt,node distance=1cm and 1.2cm,on grid,auto]
% 		\tikzstyle{state}=[circle, inner sep=0mm, outer sep=0mm, minimum size=6mm, draw=black];
% 		%\clip (-2,1.5) rectangle (2,-0.4);
% 		\node[state] (q_0) {$q$}; 
% 		\node[state] (q_1) [above right=of q_0] {$q^{0,0}$}; 
% 		\node[state] (q_2) [right=of q_0] {$q^{0,1}$}; 
% 		\node[state] (q_3) [above left=of q_0] {$q^{1,0}$}; 
% 		\node[state] (q_4) [left=of q_0] {$q^{1,1}$}; 
		
% 		\path[->] 
% 		(q_0)
% 		 edge node [pos=0.6] {$0$} (q_1)
% 		 edge node [pos=0.6] {$0$} (q_2)
% 		 edge node [pos=0.6, swap] {$1$} (q_3)
% 		 edge node [pos=0.6, swap] {$1$} (q_4);
% 	\end{tikzpicture}%
% 	%\hspace{1cm}%
% 	\hspace{0.5cm}%
% 	\begin{tikzpicture}[shorten >=1pt,node distance=1cm and 1.2cm,on grid,auto]
% 	    \tikzstyle{state}=[circle, inner sep=0mm, outer sep=0mm, minimum size=6mm, draw=black];
% 	    %\clip (-3.5,2) rectangle (3.5,-0.4);
% 		\node[state] (q_0) {$q$}; 
% 		\node[state, label={[font=\footnotesize,label distance=-.1cm]above:$q_a$}] (q_1) [above right=of q_0] {$\color{red} \top$}; 
% 		\node[state, label={[font=\footnotesize,label distance=-.1cm]above:$q_b$}] (q_2) [right=of q_0] {$\color{red} \top$}; 
% 		\node[state, label={[font=\footnotesize,label distance=-.1cm]above:$q_c$}] (q_3) [above left=of q_0] {$\color{red} \top$}; 
% 		\node[state, label={[font=\footnotesize,label distance=-.1cm]above:$q_d$}] (q_4) [left=of q_0] {$\color{red} \top$}; 
% 		\node[state, label={[font=\footnotesize,label distance=-.1cm]above:$q^{0,0}$}] (q_1b)[right=of q_1] {$\color{red} \top$}; 
% 		\node[state, label={[font=\footnotesize,label distance=-.1cm]above:$q^{0,1}$}] (q_2b)[right=of q_2] {$\color{red} \top$}; 
% 		\node[state, label={[font=\footnotesize,label distance=-.1cm]above:$q^{1,0}$}] (q_3b)[left=of q_3] {$\color{red} \top$}; 
% 		\node[state, label={[font=\footnotesize,label distance=-.1cm]above:$q^{1,1}$}] (q_4b)[left=of q_4] {$\color{red} \top$}; 
		
% 		\path[->] 
% 		(q_0)
% 		 edge node [pos=0.6] {$a$} (q_1)
% 		 edge node [pos=0.6] {$b$} (q_2)
% 		 edge node [pos=0.6, swap] {$c$} (q_3)
% 		 edge node [pos=0.6, swap] {$d$} (q_4)
% 		(q_1) edge node [pos=0.6] {$b$} (q_1b)
% 		(q_2) edge node [pos=0.6] {$a$} (q_2b)
% 		(q_3) edge node [pos=0.6, swap] {$d$} (q_3b)
% 		(q_4) edge node [pos=0.6, swap] {$c$} (q_4b);
% 	\end{tikzpicture}
%  	\caption{Every state and its 4 transitions in $\cN$ (left) turn into 8 transitions in $\cT_2$ (right). All transitions not drawn in the right figure lead to $q_\bot$, a sink state labelled $\color{red}\bot$.}
%  	\label{fig:PSPACE_reduction_T2}
%  \end{minipage}
% \end{figure}
%Assume $\Sigma^* = L(A)$, then all words have a path. Let $x\in\Lambda$. It is straightforward to see that $T_1(x)=\mathbf{o}_1^{|x|}$. Moreover, $T_2(x')=\mathbf{o}_1^{|x|}$, where $x'$ is obtained as follows: Replace every appearance of $ab$ with $0$ and of $cd$ with $1$, and call the resulting word $\hat{x}$. This word has a generating path in $A$, denoted by $s_1\rightarrow s_2\rightarrow\cdots\rightarrow s_k$. Now, every path in $A$ has a matching path in $T_1$ (in a one-to-one manner).

We now claim that $L(\cN)=\{0,1\}^*$ iff $\cT_1\equiv_{2,\fL}\cT_2$.
For the first direction, assume $L(\cN)=\{0,1\}^*$. Observe that $\cT_2\prec_{2,\fL}\cT_1$ independently: for every $x\in (ab+cd)^*$, denote $\cT_2(x)=\top^{m}\bot^{|x|-m}$, then we can construct $x'\req[2]x$ such that $\cT_1(x')=\top^{m}\bot^{|x|-m}$ by leaving $x$ unchanged $m$ steps, and then permuting the letters such that the run of $\cT_1$ moves to the sink labelled $\bot$ (indeed, observe that $m$ must be even by the construction of $\cT_2$, and hence $\cT_1$ can permute e.g., $ab$ to $ba$ in order to start outputting $\bot$ on an even step).

Next, we show that $\cT_1\prec_{2,\fL}\cT_2$. Consider $x\in (ab+cd)^*$, so that $\cT_1(x)=\top^{|x|}$, and let $w\in \{0,1\}^*$ be the word obtained from $x$ by identifying $ab$ with $0$ and $cd$ with $1$. Since $L(\cN)=\{0,1\}^*$, there exists a run (and hence an accepting run) of $\cN$ on $w$, denoted $s_0,s_1,\ldots,s_n$. We now obtain $x''\req[2]x$ by identifying each letter $0$ in $x$ with either $ab$ or $ba$, and each letter $1$ with $cd$ or $dc$, such that the run of $\cT_2$ on $x''$ simulates the run of $\cN$ on $w$. Thus, $\cT_2(x'')=\top^{|x''|}$, and $\cT_2(x'')\req[2]\cT_1(x)$, so we are done.

Conversely, if $\cT_1\equiv_{2,\fL}\cT_2$, then in particular $\cT_1\prec_{2,\fL}\cT_2$. We claim that $L(\cN)=\{0,1\}^*$. Consider $w\in \{0,1\}^*$. Dually to the above, we obtain from $w$ a word $x\in (ab+cd)^*$ by identifying $0$ with $ab$ and $1$ with $cd$, so that $\cT_1(x)=\top^{|x|}$. 
Since $\cT_1\prec_{2,\fL}\cT_2$, there exists $x'\req[2]x$ such that $\cT_2(x')=\top^{|x|}$. Observe that $x'$ must be obtained from $x$ by (possibly) changing each $ab$ to $ba$ and each $cd$ to $dc$. In particular, the run of $\cT_2$ on $x'$ induces a run of $\cN$ on $w$ by identifying both $ab$ and $ba$ as 0 and both $cd$ and $dc$ as 1. This gives $w\in L(\cN)$, so $L(\cN)=\{0,1\}^*$, which concludes the proof. \qed

\section{Proof of Theorem~\ref{thm:existential_equivalence_PSPACE-H}}
\label{apx:proof_existential_PSPACE-H}

In order to show that existential round equivalence is $\PSPACE$-hard, we build upon the reduction in the proof of Theorem~\ref{thm:equivalence_PSPACE-H}: we again show a reduction from the universality problem for \glspl{nfa} over alphabet $\{0,1\}$ where all states are accepting and the degree of nondeterminism is at most 2 (cf.~\cref{lem:universalityofnfa}).

Consider an \gls{nfa} $\cN=\tup{Q,\{0,1\},\delta,q_0,Q}$ where $|\delta(q,\sigma)|\le 2$ for every $q\in Q$ and $\sigma\in \{0,1\}$.
We construct two transducers $\cT_1$ and $\cT_2$ over input and output alphabets $\sI=\{a,b,c,d,\#\}$ and $\sO=\{\top,\bot\}$ and $\fL\subseteq \sI^*$, such that $L(\cN)=\{0,1\}^*$ iff $\cT_1\equiv_{2,\fL}\cT_2$. 

Intuitively, the idea is to use a similar encoding of $\{0,1\}$ in $\{a,b,c,d\}$ whereby $0$ corresponds to either $ab$ or $ba$ and $1$ to $cd$ or $dc$. Now, however, since $k$ is not fixed to $2$, we also allow arbitrary padding with sequences of $\#\#$.

Set $\fL=(ab+cd+\#\#)^*$ (given as a 5 state \gls{dfa}). We construct $\cT_1$ and $\cT_2$ similarly to the proof of~\cref{thm:equivalence_PSPACE-H}, by adding self-cycles of length 2 upon reading $\#\#$, from every state except the sink $q_\bot$. See~\cref{fig:existential_PSPACE_reduction_T1,fig:existential_PSPACE_reduction_T2} for an illustration.

\begin{figure}[ht]
	\centering
	\begin{tikzpicture}[shorten >=1pt,node distance=1.5cm and 1.6cm,on grid,auto]
	\tikzstyle{smallnode}=[circle, inner sep=0mm, outer sep=0mm, minimum size=8mm, draw=black];
		\node[smallnode] [initial text={}] (q_0) [initial above] {$\color{red} \top$}; 
		\node[smallnode] (q_1) [right=of q_0] {$\color{red} \top$}; 
		\node[smallnode] (q_2) [left=of q_0] {$\color{red} \top$};
		\node[smallnode] (q_3) [below=of q_0] {$\color{red} \top$};
		\node[smallnode] (q_4) [below=of q_3] {$\color{red} \bot$};
		
		\path[->] 
		(q_0)
		 edge [bend right, swap, pos=0.6] node {$a$} (q_1)
		 edge [bend left, pos=0.6] node {$c$} (q_2)
		 edge [bend left=10] node {$\#$} (q_3)
		 edge [out=-135, in=135, swap, pos=0.7] node {$b,d$} (q_4)
		(q_1) 
		 edge [bend right, swap] node {$b$} (q_0)
		 edge [out=-45, in=-45] node {$a,c,d,\#$} (q_4)
		(q_2)
	     edge [bend left] node {$d$} (q_0)
		 edge [out=-135, in=-135, swap] node {$a,b,c,\#$} (q_4)
		(q_3)
		 edge [bend left=10] node {$\#$} (q_0)
		 edge [] node {$a,b,c,d$} (q_4);
		 
	\end{tikzpicture}
	\caption{The transducer $\cT_1$ in  the proof of~\cref{thm:existential_equivalence_PSPACE-H}.}
	\label{fig:existential_PSPACE_reduction_T1}
\end{figure}

\begin{figure}[ht]
	\centering
	\begin{tikzpicture}[shorten >=1pt,node distance=1.5cm and 1.6cm,on grid,auto]
		\tikzstyle{state}=[circle, inner sep=.5mm, outer sep=0mm, minimum size=8mm, draw=black];
		\node[state] (q_0) {$q$}; 
		\node[state] (q_1) [above right=of q_0] {$q^{0,0}$}; 
		\node[state] (q_2) [right=of q_0] {$q^{0,1}$}; 
		\node[state] (q_3) [above left=of q_0] {$q^{1,0}$}; 
		\node[state] (q_4) [left=of q_0] {$q^{1,1}$}; 
		
		\path[->] 
		(q_0)
		 edge node [pos=0.6] {$0$} (q_1)
		 edge node [pos=0.6] {$0$} (q_2)
		 edge node [pos=0.6,swap] {$1$} (q_3)
		 edge node [pos=0.6,swap] {$1$} (q_4);
	\end{tikzpicture}%
	\hspace{2cm}%
	\begin{tikzpicture}[shorten >=1pt,node distance=1.5cm and 1.6cm,on grid,auto]
	    \tikzstyle{state}=[circle, inner sep=.5mm, outer sep=0mm, minimum size=8mm, draw=black];
		\node[state] (q_0) {$q$}; 
		\node[state] (q') [above=of q_0] {$q_\#$}; 
		\node[state, label={[font=\small]above:$q_a$}] (q_1) [above right=of q_0] {$\color{red} \top$}; 
		\node[state, label={[font=\small]above:$q_b$}] (q_2) [right=of q_0] {$\color{red} \top$}; 
		\node[state, label={[font=\small]above:$q_c$}] (q_3) [above left=of q_0] {$\color{red} \top$}; 
		\node[state, label={[font=\small]above:$q_d$}] (q_4) [left=of q_0] {$\color{red} \top$}; 
		\node[state, label={[font=\small]right:$q^{0,0}$}] (q_1b)[right=of q_1] {$\color{red} \top$}; 
		\node[state, label={[font=\small]right:$q^{0,1}$}] (q_2b)[right=of q_2] {$\color{red} \top$}; 
		\node[state, label={[font=\small]left:$q^{1,0}$}] (q_3b)[left=of q_3] {$\color{red} \top$}; 
		\node[state, label={[font=\small]left:$q^{1,1}$}] (q_4b)[left=of q_4] {$\color{red} \top$}; 
		
		\path[->] 
		(q_0)
		 edge node [pos=0.8] {$a$} (q_1)
		 edge node [pos=0.8] {$b$} (q_2)
		 edge node [pos=0.8, swap] {$c$} (q_3)
		 edge node [pos=0.8, swap] {$d$} (q_4)
		 edge [bend left=10, pos=0.6] node {$\#$} (q')
		(q_1) edge node [pos=0.6] {$b$} (q_1b)
		(q_2) edge node [pos=0.6] {$a$} (q_2b)
		(q_3) edge node [pos=0.6, swap] {$d$} (q_3b)
		(q_4) edge node [pos=0.6, swap] {$c$} (q_4b)
		(q') edge [bend left=10, pos=0.5] node {$\#$} (q_0);
	\end{tikzpicture}
	\caption{Every state and its 4 transitions in $\cN$ (left) turn into 10 transitions in $\cT_2$ (right). All transitions not drawn in the right figure lead to $q_\bot$, a sink state labelled $\color{red}\bot$.}
	\label{fig:existential_PSPACE_reduction_T2}
\end{figure}

We claim that $L(\cN)=\{0,1\}^*$ iff there exists $k>0$ such that $\cT_1\equiv_{k,\fL}\cT_2$.
For the first direction, assume $L(\cN)=\{0,1\}^*$, then we can show that $\cT_1\equiv_{2,\fL}\cT_2$ by following the proof of~\cref{thm:equivalence_PSPACE-H} line for line, with the addition that blocks of the form $\#\#$ leave the state of both $\cT_1$ and $\cT_2$ unchanged.

For the converse direction, assume $\cT_1\equiv_{k,\fL}\cT_2$, and in fact we only assume $\cT_1\prec_{k,\fL}\cT_2$ for some $k>0$. We further assume w.l.o.g.\! that $k$ is even, otherwise we can just take $2k$ (since we also have $\cT_1\prec_{2k,\fL}\cT_2$).

Consider $w\in \{0,1\}^*$. We obtain from $w$ a word $x\in (ab+cd+\#\#)^*$ by identifying $0$ with $ab\#^{k-2}$ and $1$ with $cd\#^{k-2}$. Observe that $\cT_1(x)=\top^{|x|}$, and that $x$ is indeed a $k$-round word in $\fL$, with each round being either $ab\#^{k-2}$ or $cd\#^{k-2}$. 

Since $\cT_1\prec_{k,\fL}\cT_2$, there exists $x'\req[k]x$ such that $\cT_2(x')=\top^{|x|}$. Observe that $x'$ must be obtained from $x$ by (possibly) changing each $ab$ to $ba$ and each $cd$ to $dc$, and by shifting the location of this pair within the $\#$ symbols. Indeed, otherwise the run of $\cT_2$ on $x'$ ends in $q_{\bot}$.
In particular, the run of $\cT_2$ on $x'$ induces a run of $\cN$ on $w$ by identifying both $ab$ and $ba$ as 0 and both $cd$ and $dc$ as 1. Thus, $w\in L(\cN)$, so $L(\cN)=\{0,1\}^*$, and the proof is concluded. \qed

% \begin{section}
% This a former try of proving existential. Either Appendix or drop.

% \begin{hypothesis}
% Let $A,B$ be two \glspl{nfa} over $\Sigma$. There exists a bound $M=M(A,B)$ that satisfies the following: let $x\in \Sigma^\star$ and $x'\equiv x$ a permutation thereof. Denote $q'\in\delta_1(q,x)$ and $s'\in\delta_2(s,x')$, then there exists such an $x$ with $|x|<M$.
% \end{hypothesis}

% To understand how this theorem can help our cause, recall that the algorithm for checking $k$-simulation in rounds was based on a construction of a \gls{dfa} $A_1^k$ and an \gls{nfa} $A_2^k$ so that, if we fix $T_1$ and $T_2$ and gradually increase $k$, we will have a sequence of automata with identical states, wherein only the transitions and the alphabet change.

% If we succeed to show that for some $k_1$ and $k_2$ the two pairs of automata are identical in transitions, then by help of the theorem we can bound the difference $k_2-k_1$ and hopefully get a deterministic analysis of the values of $k$ which would work, i.e. we would solve the problem of bounding $k$.

% \end{section}
