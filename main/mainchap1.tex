\chapter{Round Simulation and Round Equivalence}
\label{chap:round_equivalence}

Consider two $k$-round words $x,y\in \Sigma^{kR}$ with the same number of rounds $R$, and denote their rounds by $x=\alpha_0\cdots \alpha_{R-1}$ and $y=\beta_0\cdots \beta_{R-1}$. We say that $x$ and $y$ are \emph{$k$-round equivalent}, denoted $x\req[k] y$ (or $x\req y$, when $k$ is clear from context)\footnote{Conveniently, our symbol for round equivalence is a rounded equivalence.}, if for every $0\le r<R$ we have that $\fP(\alpha_r)=\fP(\beta_r)$. That is, $x\req y$ iff the $r$-th round of $y$ is a permutation of the $r$-th round of $x$, for every $r$. Indeed, $\req$ is an equivalence relation.

\begin{example}[Round-equivalence for words]
\label{example:word-req}
Consider the words $x=abaabbabbbaa$ and $y=baabbaabbaba$ over the alphabet $\Sigma=\{a,b\}$. Looking at the words as $3$-round words, one can see in~\cref{tab:word-req-3} that rounds of length 3 in $y$ are all permutations of those in $x$, which gives $x\req[3] y$. However, looking at the rounds of length 4 of $x,y$, the number of occurrences of $b$ already in the first round of $x$ and of $y$ is different, so $x\not\req[4] y$, as illustrated in~\cref{tab:word-req-4}.
\begin{table}[!htb]
\hspace{.05\linewidth}%
\begin{minipage}{.4\linewidth}
    \centering
    \caption{$x$ and $y$ are 3-round equivalent}
    \vspace{2mm}
    \begin{tabular}{c||c|c|c|c}
        $x$ & aba & abb & abb & baa \\
        \hline
        $y$ & baa & bba & abb & aba \\
    \end{tabular}
    \label{tab:word-req-3}
\end{minipage}%
\hspace{.1\linewidth}%
\begin{minipage}{.4\linewidth}
    \centering
    \caption{$x$ and $y$ are not 4-round equivalent}
    \vspace{2mm}
    \begin{tabular}{c||c|c|c}
        $x$ & \emph{abaa} & bbab & bbaa \\
        \hline
        $y$ & \emph{baab} & baab & baba \\
    \end{tabular}
    \label{tab:word-req-4}
\end{minipage}
\end{table}
\end{example}

Let $\sI$ and $\sO$ be input and output alphabets, let $\fL\subseteq \sI^*$ be a regular language, and let $k>0$. Consider two $\sI/\sO$ transducers $\cT_1$ and $\cT_2$. We say that \emph{$\cT_2$ $k$-round simulates $\cT_1$ restricted to $\fL$}, denoted $\cT_1\prec_{k,\fL} \cT_2$, if for every $k$-round word $x\in \fL$ there exists a $k$-round word $x'\in \sI^*$ such that $x\req[k] x'$ and $\cT_1(x)\req[k] \cT_2(x')$. 

Intuitively, $\cT_1\prec_{k,\fL} \cT_2$ if for every input word $x\in \fL$, we can permute each round of length $k$ in $x$ to obtain a new word $x'$, such that the outputs of $\cT_1$ on $x$ and of $\cT_2$ on $x'$ are $k$-round equivalent.
Note that the definition is not symmetric: the input $x$ for $\cT_1$ is universally quantified, while $x'$ is chosen according to $x$. We illustrate this in~\cref{example:def-asymmetric}.

If $\cT_1\prec_{k,\fL} \cT_2$ and $\cT_2\prec_{k,\fL} \cT_1$ we say that $\cT_1$ and $\cT_2$ are \emph{$k$-round equivalent restricted to $\fL$}, denoted $\cT_1\equiv_{k,\fL} \cT_2$. 
In the special case where $\fL=\sI^*$ (i.e., when we require the simulation to hold for every input), we omit it from the subscript and write $\cT_1\prec_k \cT_2$.

\begin{example}[Round Robin]
\label{example:transducer-req}
We consider a simple version of the \gls{rr} scheduler for three processes $\cP=\{0,1,2\}$. In each time step, the scheduler outputs either a singleton set containing the ID of the process whose request is granted, or an empty set if the process whose turn it is did not make a request. 
Depending on the ID $i\in \{0,1,2\}$ of the first process, we model the scheduler as a $2^{\cP}/2^{\cP}$ transducer $\cT_i = \tup{2^\cP, 2^\cP, Q, q_{(i-1)\%3}, \delta, \lab}$ depicted in~\cref{fig:RR}, where \% is the $\bmod$ operator, $Q=\left\{q_0,q_1,q_2,q'_0,q'_1,q'_2\right\}$, $\delta(q_i, \sigma)=q_{(i+1)\%3}$ if $i+1\in\sigma$ and $\delta(q_i,\sigma)=q'_{(i+1)\%3}$ otherwise, $\lab(q_i)=\{i\}$ and $\lab(q'_i)=\emptyset$.
\begin{figure}[ht]
	\centering
	\small
	\begin{tikzpicture}[shorten >=1pt,node distance=1.5cm and 4cm,on grid,auto]
	\tikzset{state/.style={draw,ellipse,minimum size=0pt}};
		\clip (-1.5,1.5) rectangle (9, -3);
		\node[state] (q_0) {$q_0/{\color{red}\{0\}}$}; 
		\node[state] (q_1) [right=of q_0] {$q_1/{\color{red}\{1\}}$}; 
		\node[state] (q_2) [right=of q_1] {$q_2/{\color{red}\{2\}}$}; 
		\node[state] (q'_0) [below=of q_0] {$q'_0/{\color{red}\emptyset}$}; 
		\node[state] (q'_1) [below=of q_1] {$q'_1/{\color{red}\emptyset}$}; 
		\node[state] (q'_2) [below=of q_2] {$q'_2/{\color{red}\emptyset}$}; 
		\path[->] 
		(q_0)  edge node {$1$} (q_1)
		edge [sloped] node[pos=0.7] {$\lnot 1$} (q'_1)
		(q_1)  edge node {$2$} (q_2)
		edge [sloped] node [pos=0.7] {$\lnot 2$} (q'_2)
		(q_2)  edge [bend right=15, above] node {$0$} (q_0)
		edge [sloped, out=160, in=130, looseness=1.5, above] node  [below,pos=0.5] {$\lnot 0$} (q'_0)
		(q'_0) edge [sloped] node [pos=0.7] {$1$} (q_1)
		edge node {$\lnot 1$} (q'_1)
		(q'_1) edge [sloped] node [pos=0.7] {$2$} (q_2)
		edge node {$\lnot 2$} (q'_2)
		(q'_2) edge [sloped, out=200, in=230, looseness=1.5, below] node [above,pos=0.5] {$0$} (q_0)
		edge [bend left=15] node {$\lnot 0$} (q'_0)
		(q_2); 
	\end{tikzpicture}
	\caption{The transducer $\cT_i$ for RR, initial state omitted. The input letters $\sigma$ and $\lnot\sigma$ mean all letters from $2^\cP$ that, respectively, contain or do not contain $\sigma$. The labels are written in red.}
	\label{fig:RR}
\end{figure}

Technically, the initial state changes the behaviour of $\cT_i$ significantly (e.g. we have $\cT_0(\{0\}\{2\}\{1\})=\{0\}\emptyset\emptyset$ whereas $\cT_1(\{0\}\{2\}\{1\})=\emptyset\{2\}\emptyset$). Conceptually, however, changing the initial state does not alter the behaviour, as long as the requests are permuted accordingly. This is captured by round equivalence, as follows.

We argue that, if we allow permutation of the input letters, then the set of processes whose requests are granted in each round is independent of the start state. This is equivalent to saying $\cT_0 \equiv_k \cT_j$ for $j\in \{1,2\}$, which indeed holds: if $j=1$ then we permute all rounds of the form $\sigma_0 \sigma_1 \sigma_2$ to $\sigma_1 \sigma_2 \sigma_0$, and similarly if $j=2$ then we permute all rounds to $\sigma_2 \sigma_0 \sigma_1$. It is easy to see that the run of $\cT_i$ on the permuted input grants outputs that $k$-round equivalent to the output of $\cT_0$ on the non-permuted input.
\end{example}

In~\cref{example:transducer-req}, the transducers satisfied not only round simulation, but also round equivalence. We now show that this is not always the case for simulating transducers.

\begin{example}[Round simulation is not symmetric]
\label{example:def-asymmetric}
Consider the $\sI/\sO$ transducers $\cT_1$ and $\cT_2$ over the alphabet $\sI=\{a,b\}$ and $\sO=\{0,1\}$, depicted in~\cref{fig:def-asymmetry}. 
\begin{figure}[ht]
	\centering
	\begin{tikzpicture}[shorten >=1pt,node distance=1.1cm and 2cm,on grid,auto,initial text = {}]
	\clip (-0.9,0.55) rectangle (4.5, -1.5);
	    \tikzset{every state/.style={minimum size=0pt}}
		\node[state,initial] (q_0) {\color{red} $1$}; 
		\node[state] (q_b) [right=of q_0] {\color{red} $0$}; 
		\node[state] (q_a) [below=of q_0] {\color{red} $0$}; 
		\node[state] (q_sink) [below=of q_b] {\color{red} $1$}; 
		\node[state] (q_bb) [right=of q_b] {\color{red} $1$}; 
		\path[->] 
		(q_0)  edge [bend left=15] node {$b$} (q_b)
		edge [bend left=15] node {$a$} (q_a)
		(q_b)  edge [bend left=15] node {$a$} (q_0)
		edge node {$b$} (q_bb)
		(q_bb) edge node[pos=0.3] {$a,b$} (q_sink)
		(q_a)  edge node[below] {$a$} (q_sink)
		edge [bend left=15] node {$b$} (q_0)
		(q_sink) edge [loop right] node {$a,b$} ();
	\end{tikzpicture}%
	\hspace{2cm}%
	\begin{tikzpicture}[shorten >=1pt,node distance=1.1cm and 2cm,on grid,auto,initial text = {}]
	    \clip (-0.9,0.55) rectangle (4.5, -1.5);
	    \tikzset{every state/.style={minimum size=0pt}}
		\node[state,initial] (q_0) {\color{red} $0$}; 
		\node[state] (q_b) [right=of q_0] {\color{red} $1$}; 
		\node[state] (q_a) [below=of q_0] {\color{red} $0$}; 
		\node[state] (q_sink) [below=of q_b] {\color{red} $1$}; 
		\node[state] (q_bb) [right=of q_b] {\color{red} $0$}; 
		\path[->] 
		(q_0)  edge [bend left=15] node {$b$} (q_b)
		edge [bend left=15] node {$a$} (q_a)
		(q_b)  edge [bend left=15] node {$a$} (q_0)
		edge node {$b$} (q_bb)
		(q_bb) edge node[pos=0.3] {$a,b$} (q_sink)
		(q_a)  edge node[below] {$a$} (q_sink)
		edge [bend left=15] node {$b$} (q_0)
		(q_sink) edge [loop right] node {$a,b$} ();
	\end{tikzpicture}%
	\caption{Transducers $\cT_1$ (left) and $\cT_2$ (right) illustrate the asymmetry in the definition of round equivalence (see~\cref{example:def-asymmetric}).}
	\label{fig:def-asymmetry}
\end{figure}
We claim that $\cT_1\prec_2 \cT_2$ but $\cT_2\not\prec_2 \cT_1$. Starting with the latter, observe that $\cT_2(ab)=00$, but $\cT_1(ab)=\cT_1(ba)=01$. Since $00\not\req[2]01$, we have $\cT_2\not\prec_2 \cT_1$. 

We turn to show that $\cT_1\prec_{2}\cT_2$.  Observe that for every input word of the form $x\in (ab+ba)^m$, we have $\cT_1(x)=(01)^m$, and $x\req[2] (ba)^m$. So in this case we have that $\cT_2((ba)^m)=(10)^m\req[2] (01)^m$. Next, for $x\in (ab+ba)^m\cdot bb\cdot w$ for some $w\in \sI^*$ we have $\cT_1(x)=(01)^m 01 1^{|w|}$ and $x\req[2] (ba)^m\cdot bb\cdot w$, for which $\cT_2((ba)^m\cdot bb\cdot w)=(01)^m 10 1^{|w|}\req[2]\cT_1(x)$.
The case where $x\in (ab+ba)^m\cdot aa\cdot w$ is handled similarly. We conclude that $\cT_1\prec_{2} \cT_2$.
\end{example}

Round simulation and round equivalence give rise to the following decision problems: 
\begin{itemize}
	\item In \emph{fixed round simulation} (resp. \emph{fixed round equivalence}) we are given transducers $\cT_1,\cT_2$, an \gls{nfa} for the language $\fL$, and $k>0$ in unary, and we need to decide whether $\cT_1\prec_{k,\fL} \cT_2$ (resp. whether $\cT_1\equiv_{k,\fL}\cT_2$).
	\item In \emph{existential round simulation} (resp. \emph{existential round equivalence}) we are given transducers $\cT_1,\cT_2$ and an \gls{nfa} for the language $\fL$, and we need to decide whether there exists $k>0$ such that $\cT_1\prec_{k,\fL} \cT_2$ (resp. $\cT_1\equiv_{k,\fL}\cT_2$). 
\end{itemize}
In the following we identify $\fL$ with an \gls{nfa} (or \gls{dfa}) for it, as we do not explicitly rely on its description.

We start by showing that deciding equivalence (both fixed and existential) is reducible, in polynomial time, to the respective simulation problem.
\begin{lemma}
	\label{lem:reduce_equiv_to_simulation}
    Fixed (resp. existential) round equivalence is Turing reducible in polynomial time to fixed (resp. existential) round simulation.
\end{lemma}
\begin{proof}
First, we can clearly reduce fixed round equivalence to fixed round simulation: given an algorithm that decides, given $\cT_1,\cT_2$, $\fL$ and $k>0$, whether $\cT_1\prec_{k,\fL} \cT_2$, we can decide whether $\cT_1\equiv_{k,\fL} \cT_2$ by using it twice to decide whether both $\cT_1\prec_{k,\fL} \cT_2$ and $\cT_1\prec_{k,\fL} \cT_2$ hold. 

A slightly more careful examination shows that the same approach can be taken to reduce existential round equivalence to existential round simulation, using the following observation: if $\cT_1\prec_{k,\fL} \cT_2$, then for every $m\in \bbN$ it holds that $\cT_1\prec_{mk,\fL}\cT_2$. Indeed, we can simply group every $m$ rounds of length $k$ and treat them as a single round of length $mk$.

Now, given an algorithm that decides, given $\cT_1,\cT_2$ and $\fL$, whether there exists $k>0$ such that $\cT_1\prec_{k,\fL} \cT_2$, we can decide whether $\cT_1\equiv_{k,\fL} \cT_2$ by using the algorithm twice to decide whether there exists $k_1$ such that $\cT_1\prec_{k_1,\fL} \cT_2$ and $k_2$ such that $\cT_2\prec_{k_2,\fL} \cT_1$ hold. If there are no such $k_1,k_2$, then clearly $\cT_1 \not\equiv_{k,\fL} \cT_2$. However, if there are such $k_1,k_2$, then by the observation above we have $\cT_1\equiv_{k_1k_2,\fL}\cT_2$ (we can also take $\lcm(k_1,k_2)$ instead of $k_1k_2$).
\end{proof}
 
By~\cref{lem:reduce_equiv_to_simulation}, for the purpose of upper-bounds, we focus henceforth on round simulation.
