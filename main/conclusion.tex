\chapter{Conclusion and Open Questions}
\label{chap:conclusion}
\label{chap:discussion}
\label{chap:future}
 
In this work, we introduced round simulation and provided decision procedures and lower bounds (some with remaining gaps) for the related algorithmic problems.

Round simulation, and in particular its application to round symmetry, is only an instantiation of a more general framework of symmetry, by which we measure the stability of transducers under local changes to the input. In particular, there is place for additional notions of symmetry and simulation to be studied, and the existing ones extended. Some such variants were presented and discussed in~\cref{sec:variations_rs}. An additional possibly interesting notion of simulation is \emph{window simulation}, where we use a sliding window of size $k$ instead of disjoint $k$-rounds as in round simulation.
In addition, the setting of infinite words is of interest. Beside the notion of \emph{ultimate simulation} presented in~\cref{sec:infinite_setting}, a possible future direction is to define an analogous symmetry for the probabilistic setting, where state transitions are equipped with probabilities and every input leads to some probability distribution over the state space~\cite{Almagor2020b}.
Finally, other types of transducers may also require variants of simulation, such as streaming-string transducers~\cite{Alur2010}.

Beside extending the study of the different notions of symmetry, a few gaps have remained open along the way of this study. In terms of complexity bounds for the existential case, can we do better than the $\EEEEEXP$ bound in~\cref{rmk:complexity}?
Furthermore, regarding the simulation mapping from~\cref{chap:equiv_mapping} that maps every input word to a corresponding word for the simulating transducer, can we find a sub-Turing model that defines it? Here, too, other models might help. For this end, a possible direction is looking into streaming-string transducers and bi-machines~\cite{Muscholl}.
