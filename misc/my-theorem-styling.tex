% \usepackage[dvipsnames]{xcolor}
% \usepackage{caption}
% \usepackage{thmtools}
% \usepackage[most]{tcolorbox}

\colorlet{bulletColor}{BrickRed}
\renewcommand\bullet{\textcolor{bulletColor}{\ensuremath{\blacktriangleright}}}
\renewcommand\qedsymbol{\textcolor{bulletColor}{\ensuremath{\blacktriangleleft}}}

\makeatletter
\thm@headfont{%
  \bullet\nobreakspace%
  \sffamily\bfseries%
}

\def\th@remark{%
  \thm@headfont{%
    \bullet\nobreakspace%
    \sffamily\bfseries%
    }%
  \normalfont % body font
  \thm@preskip\topsep \divide\thm@preskip\tw@
  \thm@postskip\thm@preskip
}
\def\@endtheorem{\endtrivlist}

\renewenvironment{proof}[1][\proofname]{\par
  \pushQED{\qed}%
  \normalfont \topsep6\p@\@plus6\p@\relax
  \trivlist
  \item[\hskip\labelsep
        \sffamily\bfseries
    #1\@addpunct{.}]\ignorespaces
}{%
  \popQED\endtrivlist
}

\newtheoremstyle{claimstyle}{\topsep}{\topsep}{}{0pt}{\sffamily}{. }{5pt plus 1pt minus 1pt}%
  {\bullet \thmname{#1}\thmnumber{ #2}\thmnote{ (#3)}%
}

\theoremstyle{plain}

\newtheorem{theorem}{Theorem}
% \ifx\numberwithinsect\relax
  \@addtoreset{theorem}{chapter}
  \expandafter\def\expandafter\thetheorem\expandafter{%
    \expandafter\thechapter\expandafter\@thmcountersep\thetheorem}
% \fi
\makeatother

\newtheorem{lemma}[theorem]{Lemma}
\newtheorem{corollary}[theorem]{Corollary}
\newtheorem{proposition}[theorem]{Proposition}
\newtheorem{exercise}[theorem]{Exercise}
\newtheorem{definition}[theorem]{Definition}
\newtheorem{conjecture}[theorem]{Conjecture}
\newtheorem{observation}[theorem]{Observation}

\theoremstyle{definition}
\newtheorem{example}[theorem]{Example}

\theoremstyle{remark}
\newtheorem{note}[theorem]{Note}
\newtheorem*{note*}{Note}
\newtheorem{remark}[theorem]{Remark}
\newtheorem*{remark*}{Remark}

\theoremstyle{claimstyle}
\newtheorem{claim}[theorem]{Claim}
\newtheorem*{claim*}{Claim}

\captionsetup[table]{labelfont=bf}
\captionsetup{labelfont=bf}

%%% UNUSED BELOW %%%

\newtcolorbox{mybox}{
enhanced,
boxrule=0pt,frame hidden,
borderline west={4pt}{0pt}{green!75!black},
% colback=green!10!white,
sharp corners
}

\newenvironment{coloredtheorem}%
    {\begin{mybox}\begin{theorem}}%
    {\end{theorem}\end{mybox}}
