% Use this file to create "glossary entries" for abbreviations and acronyms,
% and other notation. The entries defined here don't necessarily have to be 
% used in the thesis (but then you have to decide whether or not to display
% the unused entries).

% For this file to compile (and the example text in the main/prelims.tex file),
% the package glossaries-extra is required. It is automatically included unless
% the noabbrevs class option is used.

% The following will alter the style for typesetting abbreviations when using 
% the \gls command. Note you can also use multiple styles by categorizing 
% abbreviations; see the documentation for the glossaries-extras package at:
% https://ctan.org/pkg/glossaries-extra
%
%\setabbreviationstyle[acronym]{long-short-sc}
%
% If you're wondering why we're setting the seemingly-redundant "notation 
% category", that's a hack discussed here:
% https://tex.stackexchange.com/q/630541/5640 

% USAGE:
% \gls{dfa} - singular
% \glspl{dfa} - plural
% \Gls - singular, first letter capitalized
% \Glspl - plural, first letter capitalized

\newacronym[category=notation-category, description={},%
    longplural={deterministic~finite~automata}]%
    {dfa}{DFA}{deterministic~finite~automaton}

\newacronym[category=notation-category, description={},%
    longplural={nondeterministic~finite~automata}]%
    {nfa}{NFA}{nondeterministic~finite~automaton}

\newacronym[category=notation-category, description={}]%
    {pa}{PA}{Presburger~arithmetic}

\newacronym[category=notation-category, description={}]%
    {rr}{RR}{Round~Robin}

\newglossaryentry{todo}{%
  type=notation,%
  category=notation-category,%
  name=todo,%
  description=Remember to fill in this table too%
}

% \newabbreviation[%
%   type=notation,%
%   category=notation-category,%
%   description=]% This abbreviation has no description; only the abbreviation and the unabbreviated form will be shown
%   {aut}{Aut}{Automorphism group}

% --------------------------------

% Commands below will control the behavior/appearance of the list of abbreviations and acronyms

% Uncomment this command to have _all_ abbreviations and acronyms defined
% in this file appear in the final list - rather than just the ones you
% use in the thesis
% \keepUnusedAbbreviations
