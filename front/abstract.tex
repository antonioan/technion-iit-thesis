% This file contains the abstract part of your thesis - in English and
% in Hebrew (within \abstractEnglish and \abstractHebrew respectively).
%
% Notes:
% - This file uses the UTF-8 character set encoding for the Hebrew
%   text not to get garbled. Keep it that way.
% - Assuming your thesis is mainly in English, Graduate School 
%   regulations mandate the following lengths for the abstracts:
%
%      Language    Min. Length   Max. Length
%     ---------------------------------------
%      English       200 words     500 words
%      Hebrew        500 words   2,000 words
%
%   so that the Hebrew abstract typically has some content from
%   the English introduction and an overview of the results, not
%   present in the English; it is not just a translation.

\abstractEnglish{
In model checking, we work toward deciding whether a system satisfies a given specification. Often, a system exhibits some type of symmetry in its structure or in its behaviour. Such symmetries can be exploited by a designer to alleviate some of the complexity of model checking, as well as to gain insight into the behaviour of the system. Thus, we want to decide whether a given system exhibits symmetry.

Symmetry is not a well-defined concept and might come in various forms, each capturing a different characteristic behaviour. In this work, we introduce a notion of semantic symmetry in transducers, and demonstrate the definitions and the behaviours it captures, as well as pose the algorithmic questions pertaining to it and their solutions.

In particular, I present the notion of simulation by rounds, whose usefulness is in that it can be applied to process symmetry.
In this setting, words are partitioned into rounds, and
% OPTION 1 %
[OPTION 1] a transducer is round simulated by another if for every input word, we can shuffle the letters within each round such that the output of the simulating transducer on the shuffled word is itself a shuffle of the output of the simulated transducer.
% OPTION 2 %
[OR, OPTION 2] a transducer $\cT_1$ is $k$-round simulated by transducer $\cT_2$ if for every input word $x$, we can permute the letters within each round in $x$, such that the output of $\cT_2$ on the permuted word is itself a permutation of the output of $\cT_1$ on $x$.
[END]
% END %
Finally, two transducers are round equivalent if they simulate each other.

We solve two main decision problems, namely whether $\cT_2$ $k$-round simulates $\cT_1$ (1) when $k$ is given as input, and (2) for an existentially quantified $k$.
We then show that the problem of round symmetry can be reduced to round simulation and solved as such.

Several more notions are then presented and discussed, including what we call Parikh round symmetry and ultimate symmetry.

Once symmetry is established, it can be exploited to facilitate the verification procedures.

We use tools and techniques from logic, algebra and automata theory.
} % end of English abstract


\abstractHebrew{

% Note that certain commands don't work that well in Hebrew "mode".
% If this happens to you, try wrapping the command within a
% \textenglish{ } - that may (or may not) help.

כאן יבוא תקציר מורחב בעברית (כאשר שפת החיבור העיקרית היא אנגלית). היקף התקציר יהיה \textenglish{1000-2000} מילים. התקציר יהווה שלמות בפני עצמו ויהיה מובן לקורא בעל ידיעות כלליות בנושא.

בית הספר ללימודי מוסמכים מנחה מספר הנחיות לגבי התקציר בעברית:
\begin{itemize}
\item על התקציר להיכתב במשפטים מקושרים שלמים.
\item בדרך-כלל אין לציין בתקציר מקורות ספרותיים וציטוטים.
\item אין להתייחס למספר של פרק, סעיף, נוסחה, ציור או טבלה שבגוף החיבור, ואין להשתמש בקיצורים, סמלים ומונחים לא מקובלים, אלא אם יש בתקציר די מקום לזיהויים.
\end{itemize}

לעתים יש בכל-זאת יש צורך לכלול פקודה הכוללת קישור פנימי או חיצוני בתוך התקציר העברי; במצבים כאלו כדאי דרך-כלל לעטוף את הפקודה היוצרת את הקישור בתוך פקודת \textenglish{\texttt{\textbackslash{}textenglish\{\}}} כדי למנוע כל מיני פורענויות בלתי-רצויות, כגון כישלון בהידור קובץ ה-\textenglish{PDF} או שימוש בגופן העברי באופן אשר עלול שלא להנעים לעין. לדוגמה: נניח שיש לנו צורך לצטט מקור ביבליוגרפי. אם נעשה זאת סתם-כך: \textenglish{\texttt{\textbackslash{}cite\{Hoeffding\}}}, נקבל: \cite{Hoeffding}; אם נעטוף את פקודת הציטוט, כך: \textenglish{\texttt{\textbackslash{}textenglish\{\textbackslash{}cite\{Hoeffding\}\}}}, נקבל \textenglish{\cite{Hoeffding}} (כפי שהציטוטים נראים גם בטקסט באנגלית).

\subsection*{\texthebrew{תת-חלק בתקציר המורחב}}

תוכן מקוצר לגבי נושא מסוים. התייחסות ל\emph{מושג} מסוים שהחיבור בוחן. וכולי וכולי.


\subsection*{\texthebrew{נקודה מעניינת לגבי העמודים בעברית}}

שימו לב כי העמודים בעברית אמורים להיות מיוצרים בסדר ה''הפוך'', הווה אומר העמוד האחרון בקובץ ה-\textenglish{PDF} הוא הכריכה העברית, לפניו השער העברי, ודפי התקציר צריכים להופיע בסדר הפוך (וכן במספור רומי, לפי נהלי הטכניון). כך אם נתבונן במספר שבתחתית עמוד זה \textenglish{---} אשר צריך להיות העמוד הראשון בתקציר-המורחב מבחינת רצף התוכן, והינו העמוד האחרון מבין עמודי התקציר-המורחב אחרון בקובץ ה-\textenglish{PDF} \textenglish{---} נמצא את המספר \textenglish{i} ...

\newpage

... ואילו עמוד זה של התקציר-המורחב בעברית \textenglish{---} שהינו העמוד השני בתקציר-המורחב מבחינת רצף התוכן, ונמצא ראשון בקובץ ה-\textenglish{PDF} \textenglish{---} ממוספר ב-\textenglish{ii}. המטרה במספור בסדר ה"הפוך" היא, שבעת ההדפסה לא יהיה צורך להפוך דפים, לשנות את סדרם וכולי \textenglish{---} רק להדפיס ולכרוך.

} % end of Hebrew abstract
