% This file contains the abstract part of your thesis - in English and
% in Hebrew (within \abstractEnglish and \abstractHebrew respectively).
%
% Notes:
% - This file uses the UTF-8 character set encoding for the Hebrew
%   text not to get garbled. Keep it that way.
% - Assuming your thesis is mainly in English, Graduate School 
%   regulations mandate the following lengths for the abstracts:
%
%      Language    Min. Length   Max. Length
%     ---------------------------------------
%      English       200 words     500 words
%      Hebrew        500 words   2,000 words
%
%   so that the Hebrew abstract typically has some content from
%   the English introduction and an overview of the results, not
%   present in the English; it is not just a translation.

\abstractEnglish{
% For a nicer bullet in itemize
\renewcommand\textbullet{\textcolor{BrickRed}{\ensuremath{\bigstar}}}

Model checking is a verification paradigm for systems, where we are given a system and a specification and we check whether every possible computation of the system satisfies the specification.

Often, systems over multiple processes exhibit some type of symmetry in their structure or their behaviour. Symmetry is also commonly manifested in the specifications of multiple process systems.
When such symmetries are present in the system or the specification, they can be exploited by the designer and the verification algorithm to alleviate some of the complexity of model checking, as well as to gain insight into the behaviour of the system. For instance, symmetric systems enable the designer to use only representative specifications where iteration over the process identities was formerly needed.
Thus, we want to decide whether a given system or specification exhibits symmetry.

Symmetry is not a well-defined concept and might come in various forms, each capturing a different characteristic behaviour. In this work, we focus on \emph{process symmetry}, where every process $j$ has a corresponding input and output signal $i_j$ and $o_j$, and the input and output alphabets of the model are $\tI$ and $\tO$ respectively.
Process symmetry addresses the scenario where the identities of the processes may be scrambled -- that is, permuted.
For example, if the input $\{i_1,i_2\}$ is generated, the system might actually receive an input $\{i_7,i_4\}$.
Then, a system exhibits process symmetry if, intuitively, its outputs are permuted in a similar way to the inputs.
Unfortunately, deterministic systems that are process symmetric are extremely naive, as process symmetry is too restrictive for them.

In our setting, each of the system and the specification are modelled by a finite-state machine called a \emph{transducer}. In addition, words are partitioned into rounds, and a transducer $\cT$ is \emph{$k$-round symmetric} if for every permutation $\pi$ of the signals and for every input word $x$, we can scramble the letters within each round in $\pi(x)$ to obtain $x'$, such that the output of $\cT$ on $x'$ is itself a scramble of the output of $\cT$ on $x$. In other words, when $\cT$ is round symmetric, there is a way to scramble the permuted input, so that the resulting output is a scramble of the permuted output (i.e. the ``expected'' output in process symmetry).

Round symmetry is \emph{semantic}: it does not consider the structure, rather the behaviour of the system. Round symmetry gives rise to the following decision problems:
\begin{itemize}
	\item In \emph{fixed round symmetry} we are given a transducer $\cT$ and $k>0$, and we need to decide whether $\cT$ is $k$-round symmetric.
	\item In \emph{existential round symmetry} we are given a transducer $\cT$, and we need to decide whether there exists $k>0$ such that $\cT$ is $k$-round symmetric.
\end{itemize}

Notice that round symmetry defines a property of a transducer. The way we approach the decision problems is by first translating the definition of symmetry to a definition of a relation between two transducers, called \emph{round simulation}, then showing that round symmetry can be reduced to round simulation, we solve it as such.

A transducer $\cT_2$ \emph{$k$-round simulates} transducer $\cT_1$ if for every input word $x$, we can scramble the letters within each round in $x$, such that the output of $\cT_2$ on the scrambled word is itself a scramble of the output of $\cT_1$ on $x$.
In fact, we consider a somewhat more elaborate setting, by also allowing the inputs to $\cT_1$ to be restricted to some regular language $\fL$.

The mapping between the input words for $\cT_1$ and the scrambled input for $\cT_2$ is called the \emph{simulation mapping} and is also studied in the continuation of this work.

Round symmetry and round simulation -- the decision problems and their solutions, upper and lower bounds and the simulation mapping -- are the main contribution of this work.
Additionally, several more notions of symmetry are presented and discussed, including variations of round symmetry, and symmetry in the setting of infinite words.

We use tools and techniques from logic, algebra and automata theory.
} % end of English abstract

\abstractHebrew{
בדיקת מודלים היא פרדיגמת אימות למערכות, בה אנו פועלים להכרעה האם מערכת עומדת במפרט נתון.
% לעתים קרובות, מערכת מפגינה סוג כלשהו של סימטריה במבנה שלה או בהתנהגות שלה. סימטריות כאלה יכולות להיות מנוצלות על ידי מעצב התוכנה כדי להקל חלק מהמורכבות של בדיקת מודלים, כמו גם כדי לקבל תובנה לגבי התנהגות המערכת. לפיכך, אנו מעוניינים להחליט אם מערכת נתונה מפגינה סימטריה.
לעתים קרובות, מערכות מרובות תהליכים מפגינות סוג כלשהו של סימטריה במבנה שלהן או בהתנהגות שלהן. סימטריה מתבטאת בדרך כלל גם במפרטים של מערכות מרובות תהליכים.
כאשר קיימת סימטריה במערכת או במפרט, היא יכולה להיות מנוצלת על ידי המעצב ואלגוריתם האימות כדי להקל חלק מהמורכבות של בדיקת מודלים, כמו גם כדי לקבל תובנה לגבי התנהגות המערכת. לדוגמה, מערכות סימטריות מאפשרות למעצב להשתמש רק במפרטים מייצגים שבהם היה צורך בעבר באיטרציה על זהויות התהליכים.
לפיכך, אנו מעוניינים להחליט אם מערכת נתונה מפגינה סימטריה.

סימטריה אינה מושג מוגדר היטב ועשויה לבוא בצורות שונות, שכל אחת מהן תופסת התנהגות אופיינית אחרת.
בעבודה זו, אנו מתמקדים
ב\emph{סימטריית תהליכים},
איפה שלכל תהליך
$j$
יש אות קלט ופלט מקבילים
$i_j$ ו- $o_j$,
ואלפבית הקלט והפלט של המודל הם
$\tI$ ו- $\tO$
בהתאמה.
סימטריית תהליכים מתייחסת לתרחיש שבו זהויות התהליכים עשויות להיות מעורבבות 
\textenglish{--}
כלומר, עברו התמרה.
לדוגמה, אם הקלט
$\{i_1,i_2\}$
נוצר, המערכת עשויה לקבל למעשה קלט
$\{i_7,i_4\}$.
לאחר מכן, מערכת מפגינה סימטריית תהליכים אם, באופן אינטואיטיבי, הפלטים שלה עוברים התמרה בצורה דומה לקלטים.
לרוע המזל, מערכות דטרמיניסטיות שמפגינות סימטריית תהליכים הן תמימות ביותר, שכן סימטריית תהליכים מגבילה מדי עבורן.

% אנו מציגים מושג של סימטריה סמנטית במודל החישובי הנקרא משרן
% (\textenglish{transducer}),
% ומדגימים את ההגדרות וההתנהגויות שהוא לוכד, כמו גם מציגים את הבעיות האלגוריתמיות הרלוונטיות ואת הפתרונות שלהן.
% בפרט, אנו מציגים את הרעיון של 
% \emph{סימולציה בסיבובים}, 
% שתועלתו היא בכך שניתן ליישם אותה בתחום הסימטריה של תהליכים, למשל במערכות תזמון
% (\textenglish{schedulers}).

בהגדרה שלנו, המערכת ואף המפרט ממודלים על ידי מכונת מצבים סופית הנקראת
\emph{משרן}.
בנוסף,
% בסימטריה שאנו מגדירים, סימטריה בסיבובים, 
מילת הקלט של המשרן מחולקת לתתי מילים זרות באורך קבוע שנקראות סיבובים, ונאמר שמשרן
$\cT$
הינו
\emph{סימטרי בסיבובים של 
$k$}
אם עבור כל תמורה של אותות
$\pi$,
וכל מילת קלט 
$x$,
נוכל לערבב את האותיות בכל סיבוב ב-
$\pi(x)$
כדי לקבל $x'$,
כך שהפלט של
$\cT$
על
$x'$
הוא עצמו ערבוב של תוצאת התמורה על הפלט של
$\cT$
על
$x$.
במילים אחרות, כאשר
$\cT$
הוא סימטרי בסיבובים, יש דרך לערבב את הקלט המתומר, כך שהפלט המתקבל הוא ערבוב של הפלט המתומר
(כלומר הפלט ”הצפוי“ בסימטריית תהליכים).

סימטריה בסיבובים היא
\emph{סמנטית}
בכך שהיא לא מתחשבת במבנה, אלא בהתנהגות המערכת.
סימטריה בסיבובים מובילה לבעיות ההכרעה הבאות:
\begin{itemize}
\item
בבעיה 
\emph{הקבועה}
של סימטריה בסיבובים, 
ניתן לנו משרן
$\cT$ וקבוע $k>0$,
ועלינו להחליט אם
$\cT$ 
הוא סימטרי בסיבובים של
$k$.
\item
בבעיה 
\emph{הקיומית}
של סימטריה בסיבובים, 
ניתן לנו משרן
$\cT$,
וצריך להחליט אם קיים
$k>0$
כך ש-
$\cT$
הוא סימטרי בסיבובים של
$k$.
\end{itemize}

שימו לב שסימטריה בסיבובים מגדירה תכונה של משרן.
הדרך בה אנו ניגשים לבעיות ההחלטה היא על ידי תרגום תחילה את ההגדרה של סימטריה להגדרה של קשר בין שני משרנים, הנקרא
\emph{סימולציה בסיבובים},
ואז, לאחר שנראה כי ניתן לצמצם סימטריה בסיבובים לסימולציה בסיבובים, אנו פותרים אותה ככזו.

משרן
$\cT_2$
\emph{מסמלץ}
משרן
$\cT_1$
\emph{בסיבובים של
$k$}
אם עבור כל מילת קלט
$x$, 
נוכל לערבב את האותיות בכל סיבוב ב-
$x$,
כך שהפלט של
$\cT_2$
על המילה לאחר הערבוב היא עצמה ערבוב של הפלט של
$\cT_1$
על
$x$.
% לבסוף, שני משרנים נקראים
% \emph{שקולים בסיבובים של 
% $k$}
% אם הם מסמלצים זה את זה.
למעשה, אנו נותנים הגדרה קצת יותר משוכללת, בכך שאנו מאפשרים גם לכניסות ל-
$\cT_1$
להיות מוגבלות לשפה רגולרית נתונה
$\fL$.

בהתאם, בעיות ההכרעה הנוצרות מההגדרה של סימולציה בסיבובים הן האם 
$\cT_2$
מסמלץ את
$\cT_1$
בסיבובים של
$k$,
(1) כאשר
$k$
ניתן כקלט, ו-(2) עבור
$k$
בכמת קיומי (קרי, עבור
$k$
כלשהו).
% אנו גם מספקים חסמים תחתונים לבעיות (חלקם עם פערים שנותרו), 
% ולאחר מכן אנו מראים שניתן לצמצם את בעיית הסימטריה בסיבובים לסימולציה בסיבובים ולפתור אותה ככזו.

המיפוי בין מילות הקלט עבור
$\cT_1$
לבין הקלט המעורבב עבור
$\cT_2$
נקרא
\emph{מיפוי הסימולציה}
והוא נדון בהמשך עבודה זו.

סימטריה בסיבובים וסימולציה בסיבובים 
\textenglish{--}
בעיות ההחלטה ופתרונותיהן, הגבול העליון והתחתון ומיפוי הסימולציה
\textenglish{--}
הם התרומה העיקרית של עבודה זו.
בנוסף, מוצגים ונדונים עוד כמה מושגים של סימטריה, כולל וריאציות של סימטריה עגולה וסימטריה בהגדרה של אינסוף מילים.

\subsection*{\texthebrew{תקציר לפתרון של בעיות ההכרעה}}

נגדיר תחילה את המודל העיקרי שאיתו אנו עובדים.
\emph{משרן}
(\textenglish{transducer})
הינו מכונת מצבים הדומה לאוטומט סופי דטרמיניסטי, פרט לכך שלכל מצב מוצמדת אות פלט, וריצה של משרן על מילת קלט
$w$
מחזירה פלט באותו אורך, לפי המצבים שהמילה
$w$
עברה בהם. בניגוד לאוטומטים, אין משמעות לקבלה של מילים במשרן.

כאמור, אנו פותרים את בעיות ההכרעה של סימולציה בסיבובים, ומכאן גם נקבל פתרון עבור בעיית הסימטריה בסיבובים.

הבעיה הראשונה היא המקרה הקל מבין שניהם. אנו מתחילים לפתור אותה בהגדרת כלי שהוא אוטומט סופי אי-דטרמיניסטי שאנו מכנים סגור-תמורות 
(\textenglish{Permutation Closure}).

אוטומט סגור-תמורות מתקבל מהמשרן
$\cT$
ואורך הסיבוב 
$k$. 
בו, כל רצף של
$k$
מעברים במשרן המקורי נותן מעבר יחיד באוטומט החדש. יתר על כן, זהו מעבר שאינו מודע לסדר האותיות. זה נותן שהאוטומט למעשה עונה על התכונה שלפיה בחרנו את שמו: סגירות תחת תמורות, משמע שכל התמורות של אותו זוג מילות קלט ופלט מתנהגות באופן זהה זו לזו.

כדי לבנות אוטומט זה, השתמשנו באוטומט ביניים דטרמיניסטי בשם
\textenglish{Trace}
שבעצם מוציא את אותיות הפלט מהמצבים לקשתות הנכנסות.

בהישען על כלי זה, אנו מגיעים ללמה האומרת כי עבור משרנים נתונים
$\cT_1$, $\cT_2$
ואורך סיבוב
$k$,
סימולציה בסיבובים מתקיימת בין
$\cT_1$
ו-
$\cT_2$
אם ורק אם יש הכלה בין השפות של אוטומטי סגור-תמורות של
$\cT_1$
ושל
$\cT_2$.

אכן, אינטואיטיבית, אם נקבל איזשהו קלט עבור
$\cT_1$
והפלט המתאים לו, ההכלה הזו אומרת שיש תמורה כלשהי של הקלט כך שהפלט הוא עצמו גם תמורה של הפלט הנתון.

מכיוון שצמצמנו את הבעיה לבעיית ההכלה של אוטומטים אי-דטרמיניסטיים, תוצאה ישירה היא שהבעיה היא במחלקת הסיבוכיות
$\PSPACE$.
 למעשה, אנו גם מראים שהבעיה הינה
$\PSPACE$-שלמה.

כעת, במקרה הקיומי, אנו שואלים אם יש איזה אורך סיבוב שעבורו מתקיימת סימולציה סיבובית. אנו עושים זאת על ידי מציאת חסם עליון לאורך הסיבוב המינימלי שעבורו מתקיימת סימולציה סיבובית. זה נותן לנו חסם עליון סופי לקבוצת האורכים שאנו צריכים לבדוק מולם, מה שהופך את הבעיה לניתנת להכרעה.

גם כאן אנו משתמשים בכלים. הראשון הוא הטיפוס של אותיות באוטומטים. טיפוס של אות נותן תיאור של האופן שבו מצב המכונה משתנה כאשר אות זו נקראת. בהתחשב בסוגי האותיות באוטומט סגור-תמורות, אנו עוברים לדבר על קבוצת כל הטיפוסים שנצפו באוטומט, שאנו מכנים ב- ״פרופיל הטיפוס״ של האוטומט.

אנו מעוניינים בפרופילי הטיפוס של אוטומטי סגור-תמורות עבור אורכי סיבוב הולכים וגדלים. ישנו מספר סופי של אפשרויות לפרופילי טיפוס, עד כדי ריבוי. עוד אנו טוענים שיש אורך סיבוב כלשהו, שלאחריו לא מופיעים פרופילי טיפוס חדשים. כלומר, בהגעתנו אליו, כבר נראה את כל פרופילי הטיפוס שאנו הולכים לראות. זה לוקח אותנו לכלי הבא: אנו מוכיחים את המשפט הזה על ידי תרגום שלו לבעיה 
\emph{בחשבון פרסבורגר}, 
ופותרים אותו הודות למחקר בלוגיקה.

אורך סיבוב זה נותן לנו בעצם את המטרה שלנו, על ידי התבוננות בתוצאה הסופית הזו: אם פרופילי הטיפוס שווים לאורכי סיבוב
$k$
ו-
$k'$,
 ואם זה המקרה גם עבור
$\cT_1$
ו-
$\cT_2$,
אז ההכלה של השפות של אוטומטי סגור-תמורות של
$\cT_1$
ושל
$\cT_2$
(כלומר הבעיה של סימולציה סיבובית באורך סיבוב קבוע) מתקיימת עבור אורך סיבוב
$k$
אם ורק אם היא מתקיימת עבור
$k'$.
 כלומר, הבעיה של סימולציה סיבובית עם אורכי סיבוב
$k$
ו-
$k'$
הינה חופפת לשניהם, בכל פעם שפרופילי הטיפוס שלהם שווים. זה מסיים את ההוכחה, עד להסתייגות קטנה שטופלה היטב בעבודה.

לאורך העבודה, אנו נתייחס במיוחד למימוש פשוט של מערכת התזמון בשם 
\textenglish{Round Robin},
אשר ידועה והינה בשימוש אף במערכת ההפעלה של המחשב. מערכת זו מפגינה סימטריה בסיבובים באופן ניכר, ולכן היא משמשת דוגמה מאלפת להדגמת ההגדרות ואף למתן אינטואיציה בפתירת הבעיות השונות.

\subsection*{\texthebrew{סוגים נוספים של סימטריה במשרנים}}

גם כן מוצגים ונדונים בעבודה זו מספר מושגים נוספים, הן וריאציות של סימטריה בסיבובים והן סימטריות בסביבות אחרות. אחת הווריאציות של סימטריה בסיבובים מכונה סימטריה בסיבובים ע״פ פריך 
(\textenglish{Parikh}).
תחת וריאציה זו, כל אות קלט מהווה תת קבוצה של אותות – או מזהים לתהליכים השונים – כך שכל תהליך מזוהה עם אחד האותות, וכאשר מבצעים תמורה, מותר לנו לא רק להזיז את האותיות בכל סיבוב, אלא גם לערבב את האותות הבודדים בין האותיות בסיבוב.

כאמור, אנו גם בוחנים סימטריות בסביבות שונות ובמכונות חישוב אחרות, משרנים בעלי קלט אינסופי. עבודה עם מילים אינסופיות הינה נפוצה באימות פורמלי; זה נובע מהרעיון שלעיתים מצופה ממערכת לקרוא קלט ללא הפסקה, וסימטריה במערכות כאלה יכולה ללבוש אחת משתי צורות: היא תתפרש על פני כל הקלט שנקרא כבר, או שהיא תרחיב אל העתיד. סימטריה אולטימטיבית לוכדת סוג של סימטריה שמשיגה את המקרה האחרון. בסימטריה אולטימטיבית, עבור כל מילת קלט
$x$,
הפלט של
$\cT$
על
$\pi(x)$
זהה לתמורה של הפלט על הקלט המקורי
$\pi(\cT(x))$,
מלבד רישא סופית כלשהי.

\subsection*{\texthebrew{הצעות לעבודת המשך}}

חשוב לציין כי סימולציה בסיבובים, ובמיוחד היישום שלה על סימטריה בסיבובים, הינה רק דוגמה של מסגרת כללית יותר של סימטריה, שבאמצעותה אנו מודדים את היציבות של משרנים תחת שינויים מקומיים בקלט.
בפרט,
% אנו מקווים שמה שהתחלנו בעבודה זו מושך יותר סטודנטים וחוקרים, ושהם תורמים את חלקם בחקר מושגי סימטריה וסימולציה, ומרחיבים אותו להגדרות אחרות.
יש מקום לחקור מושגים נוספים של סימטריה וסימולציה,
ולהרחיב את הקיימים. 
כמה גרסאות כאלה הוצגו ונדונו בעבודה.
כיוון אפשרי הוא הרעיון של סימולציה בחלונות, שבה אנו משתמשים בחלון הזזה באורך
$k$
במקום סיבובים זרים, וסימטריה בסיבובים ע״פ פריך המתוארת מעלה. בנוסף, יש עניין לסביבה של מילים אינסופיות, איפה שהגדרנו סימולציה אולטימטיבית. לבסוף, סוגים אחרים של משרנים עשויים לדרוש גרסאות שונות של סימולציה, כגון משרנים הסתברותיים. 

מלבד הרחבת החקר במושגים השונים של סימטריה,
נותרו כמה פערים פתוחים לאורך מחקר זה.
פער אחד ניכר הינו הניתוח הנאיבי של הסיבוכיות והחסם התחתון לאלגוריתמים שהוצגו.
יתרה מזאת, 
הצגנו את מיפוי הסימולציה שממפה כל מילת קלט למילה מתאימה למשרן המסמלץ, ונשאלת השאלה האם נוכל למצוא מודל תת-טיורינג שמגדיר את המיפוי.

אנו משתמשים בכלים וטכניקות מעולם הלוגיקה, האלגברה, תורת המספרים ותורת האוטומטים. כלים נבחרים כוללים חשבון פרסבורגר
(\textenglish{Presburger arithmetic})
מלוגיקה, תכונות של מספרים ראשוניים מאלגברה ומשפט פריך מתורת המספרים.

% כאן יבוא תקציר מורחב בעברית (כאשר שפת החיבור העיקרית היא אנגלית). היקף התקציר יהיה \textenglish{1000-2000} מילים. התקציר יהווה שלמות בפני עצמו ויהיה מובן לקורא בעל ידיעות כלליות בנושא.

% בית הספר ללימודי מוסמכים מנחה מספר הנחיות לגבי התקציר בעברית:
% \begin{itemize}
% \item על התקציר להיכתב במשפטים מקושרים שלמים.
% \item בדרך-כלל אין לציין בתקציר מקורות ספרותיים וציטוטים.
% \item אין להתייחס למספר של פרק, סעיף, נוסחה, ציור או טבלה שבגוף החיבור, ואין להשתמש בקיצורים, סמלים ומונחים לא מקובלים, אלא אם יש בתקציר די מקום לזיהויים.
% \end{itemize}

% לעתים יש בכל-זאת יש צורך לכלול פקודה הכוללת קישור פנימי או חיצוני בתוך התקציר העברי; במצבים כאלו כדאי דרך-כלל לעטוף את הפקודה היוצרת את הקישור בתוך פקודת \textenglish{\texttt{\textbackslash{}textenglish\{\}}} כדי למנוע כל מיני פורענויות בלתי-רצויות, כגון כישלון בהידור קובץ ה-\textenglish{PDF} או שימוש בגופן העברי באופן אשר עלול שלא להנעים לעין. לדוגמה: נניח שיש לנו צורך לצטט מקור ביבליוגרפי. אם נעשה זאת סתם-כך: \textenglish{\texttt{\textbackslash{}cite\{Hoeffding\}}}, נקבל: \cite{Hoeffding}; אם נעטוף את פקודת הציטוט, כך: \textenglish{\texttt{\textbackslash{}textenglish\{\textbackslash{}cite\{Hoeffding\}\}}}, נקבל \textenglish{\cite{Hoeffding}} (כפי שהציטוטים נראים גם בטקסט באנגלית).

% \subsection*{\texthebrew{תת-חלק בתקציר המורחב}}

% תוכן מקוצר לגבי נושא מסוים. התייחסות ל\emph{מושג} מסוים שהחיבור בוחן. וכולי וכולי.


% \subsection*{\texthebrew{נקודה מעניינת לגבי העמודים בעברית}}

% שימו לב כי העמודים בעברית אמורים להיות מיוצרים בסדר ה''הפוך'', הווה אומר העמוד האחרון בקובץ ה-\textenglish{PDF} הוא הכריכה העברית, לפניו השער העברי, ודפי התקציר צריכים להופיע בסדר הפוך (וכן במספור רומי, לפי נהלי הטכניון). כך אם נתבונן במספר שבתחתית עמוד זה \textenglish{---} אשר צריך להיות העמוד הראשון בתקציר-המורחב מבחינת רצף התוכן, והינו העמוד האחרון מבין עמודי התקציר-המורחב אחרון בקובץ ה-\textenglish{PDF} \textenglish{---} נמצא את המספר \textenglish{i} ...

% \newpage

% ... ואילו עמוד זה של התקציר-המורחב בעברית \textenglish{---} שהינו העמוד השני בתקציר-המורחב מבחינת רצף התוכן, ונמצא ראשון בקובץ ה-\textenglish{PDF} \textenglish{---} ממוספר ב-\textenglish{ii}. המטרה במספור בסדר ה"הפוך" היא, שבעת ההדפסה לא יהיה צורך להפוך דפים, לשנות את סדרם וכולי \textenglish{---} רק להדפיס ולכרוך.

} % end of Hebrew abstract
