\personalAcknowledgementEnglish{
I would like to thank my advisor, my parents, my friends, etc. etc.

Add any thank-yous, acknowledgements, personal comments you wish to make here (in \texttt{personal-acks.tex}).

Note that this acknowledgements section only gets printed in the post-exam version of the thesis (i.e. if you leave out the \texttt{beforeDefense} option to your document class in \texttt{thesis.tex}.)

%%%%%%%%%%%%%%%
What follows is my seminar intro.

Spend 3-5 minutes on the general motivation, related work, and even what you feel you learned from the research, and if you found anything particularly interesting or exciting.

Before we dive into the research, I thought I’d take a few moments to reflect with you on the journey that I’d taken in the academia.

a.	When the first degree was nearing its end, I was thrilled, of course, but I was not well satisfied with what I learned. Yes, it’s an entire degree we’re talking about, of course I learned an awful lot. Still, something was missing.
b.	It was that I never had to come up with new ideas on my own. That is, clearly, we had to think and find solutions to problems in Algorithms and Data Structures, but these were problems already solved. It was less “inventing” and more “discovering.”
c.	I also had the opportunity to work in the hi-tech industry during the first degree. The problems we faced there were more challenging and original, for sure.
d.	And yet, being business-driven and working from deadline to deadline, the industry lacked an entire spectrum of problems that I had not yet touched. And so I decided to continue to masters.
e.	I would just like to note that I will be taking a break from the pure academia, as I join a research team in IBM in July, where my team and I will work on hybrid clouds, and the tools and knowledge that I gained from the masters will be surely beneficial.
f.	I’m also humbled to have my future manager [and team leader/members] here listening to my thesis seminar, I’m grateful for that.

1.	My masters journey began just alongside the Covid pandemic’s first emersion in Israel – March 2020.
2.	I remember I was yet to finish a project or two from the first degree, so for a while I had a foot in each degree – except that my actual feet were actually mostly… you know… in bed.
3.	So anyway, I began looking for the right field of study. And what a relief was it that, without tiring myself much, the first semester made it obvious to me: It was the course on Automata, Logic and Games, that filled my cup. I had enjoyed studying this course so much that I knew this was what I wanted to work on for the duration of my masters.
4.	So we met – the lecturer of the course who is Shaull, and I – and in my opinion there was immediate chemistry. We both love mathematics and theoretical computer science, we are both musicians, we both kept a grave distance from the moda of the age – that is, machine learning, of course (that was a joke) – and we both love it when proofs require nothing more than a pen and a paper.
5.	So we talked and formed a research goal. When he suggested the initial idea, he handed me a research paper he’d just finished working on, on process symmetry in the probabilistic setting, and that was the beginning of the journey.
6.	Reading research papers and, generally, learning about previous work, was such an enjoyable activity. You could see the excitement of the authors in their findings, and the amusing methods of proof that they used. If you want a real taste of the fun in research, just check Shaull’s YouTube channel!
7.	Of course, reading is nothing like making my own hands “dirty –“ so to speak – with research. Trying out things myself, seeing ideas build and crumble, that was the journey. It’s the suspiciously colored berry in a basket of berries – you don’t know if it will taste good or bad – or it’s the strangers you get to meet on the Camino.
8.	Anyway, we focused on what is called Formal Verification. In a time of rapid developments in both software and hardware systems, it has become ever more essential to check against malfunctions or misbehavior in a system to guarantee both quality and security.
9.	If we want to check the behavior of a system in all its possible configurations and states, that would be a lot to check. So attempting to study systems without having to do a full exploration is the general motivation for Formal Verification.
10.	In this seminar, I present a notion of symmetry in systems called Round Symmetry, which could be exploited to facilitate the verification of properties in the system. We generalize this notion of symmetry in such a way that we turn to study a relation between systems that we call Round Simulation.
11.	Without further ado, let us begin the thesis seminar… with a story.

}

\personalAcknowledgementHebrew{

אני רוצה להודות למנחה שלי, להוריי, לחבריי, וכו' וכו'.

אפשר להוסיף עוד תודות והערות אישיות כאן.

שים/י לב: קטע זה של תודות מודפס בפועל רק בגרסת החיבור שלאחר-הבחינה (הווה אומר רק אם הסרת את האפשרות \textenglish{\texttt{beforeDefense}} מן האפשרויות המועברות ל-\textenglish{\texttt{document class}} בקובץ \textenglish{\texttt{thesis.tex}}.)
}
